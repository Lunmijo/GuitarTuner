\subsubsection{Нормалізація бази даних}

Нормалізація схеми бази даних — процес розбиття одного відношення (таблиці в поняттях СУБД) відповідно до алгоритму нормалізації на кілька відношень на основі функціональних залежностей.

Нормальна форма визначається як сукупність вимог, яким має задовольняти відношення, з точки зору надмірності, яка потенційно може призвести до логічно помилкових результатів вибірки.

Таким чином, схема реляційної бази даних покроково, у процесі виконання відповідного алгоритму, переходить у першу, другу, третю і так далі нормальні форми. Якщо відношення відповідає критеріям n-ої нормальної форми та всіх попередніх нормальних форм, тоді вважається, що це відношення знаходиться у нормальній формі n-ого рівня.
