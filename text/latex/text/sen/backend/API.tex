\subsection{API веб-додатку}

Прикладний програмний інтерфейс — набір визначень підпрограм, протоколів взаємодії та засобів для створення програмного забезпечення. Спрощено - це набір чітко визначених методів для взаємодії різних компонентів. API надає розробнику засоби для швидкої розробки програмного забезпечення. API може бути для веб-базованих систем, операційних систем, баз даних, апаратного забезпечення, програмних бібліотек тощо.

\subsubsection{REST API}

REST — підхід до архітектури мережевих протоколів, які забезпечують доступ до інформаційних ресурсів. Був описаний і популяризований 2000 року Роєм Філдінгом, одним із творців протоколу HTTP. В основі REST закладено принципи функціонування Всесвітньої павутини і, зокрема, можливості HTTP. Філдінг розробив REST паралельно з HTTP 1.1 базуючись на попередньому протоколі HTTP 1.0.

Дані повинні передаватися у вигляді невеликої кількості стандартних форматів (наприклад, HTML, XML, JSON). Будь-який REST протокол (HTTP в тому числі) повинен підтримувати кешування, не повинен залежати від мережевого прошарку, не повинен зберігати інформації про стан між парами «запит-відповідь». Стверджується, що такий підхід забезпечує масштабовність системи і дозволяє їй еволюціонувати з новими вимогами. Ці особливості сприяють використанню REST API при проектуванні мікросервісних додатків \cite[158]{кучер2018мікросервісна}.

REST, як і кожен архітектурний стиль відповідає ряду архітектурних обмежень (англ. architectural constraints). Це гібридний стиль який успадковує обмеження з інших архітектурних стилів.

\paragraph{Клієнт-сервер}

Перша архітектура від якої він успадковує обмеження — це клієнт-серверна архітектура. Її обмеження вимагає розділення відповідальності між компонентами, які займаються зберіганням та оновленням даних (сервером), і тими компонентами, які займаються відображенням даних на інтерфейсі користувача та реагування на дії з цим інтерфейсом (клієнтом). Таке розділення дозволяє компонентам еволюціонувати незалежно.

\paragraph{Відсутність стану}

Наступним обмеженням є те, що взаємодії між сервером та клієнтом не мають стану, тобто кожен запит містить всю необхідну інформацію для його обробки, і не покладається на те, що сервер знає щось з попереднього запиту.

Відсутність стану не означає що стану немає. Відсутність стану означає, що сервер не знає про стан клієнта. Коли клієнт, наприклад, запитує головну сторінку сайту, сервер відповідає на запитання і забуває про клієнта. Клієнт може залишити сторінку відкритою протягом кількох років, перш ніж натиснути посилання, і тоді сервер відповість на інший запит. Тим часом сервер може відповідати на запити інших клієнтів, або нічого не робити — для клієнта це не має значення.

Таким чином, наприклад дані про стан сесії (користувача, який автентифікувався) зберігаються на клієнті, і передаються з кожним запитом. Це покращує масштабовність, бо сервер після закінчення обробки запиту може звільнити всі ресурси, задіяні для цієї операції, без жодного ризику втратити цінну інформацію. Також спрощується моніторинг і зневадження, бо для того аби розібратись, що відбувається в певному запиті, досить подивитись лише на той один запит. Збільшується надійність, бо помилка в одному запиті не зачіпає інші.

\subsubsection{SOAP API}

SOAP — протокол обміну структурованими повідомленнями в розподілених обчислювальних системах, базується на форматі XML.

Спочатку SOAP призначався, в основному, для реалізації віддаленого виклику процедур (RPC), а назва була абревіатурою: Simple Object Access Protocol — простий протокол доступу до об'єктів. Зараз протокол використовується для обміну повідомленнями в форматі XML, а не тільки для виклику процедур. SOAP є розширенням мови XML-RPC.

SOAP можна використовувати з будь-яким протоколом прикладного рівня: SMTP, FTP, HTTP та інші. Проте його взаємодія з кожним із цих протоколів має свої особливості, які потрібно відзначити окремо. Найчастіше SOAP використовується разом з HTTP.

SOAP є одним зі стандартів, на яких ґрунтується технологія веб-сервісів.

\subsubsection{JSON Web Token} \label{subsubsection:jwt}

\addCodeAsImg{\begin{umlstyle}

\umlactor[x=0, y=0, fill=blue!1]{unreg}
\umlactor[x=0, y=-3, fill=green!30]{user}
\umlactor[x=14, y=-3, fill=red!30]{admin}

\begin{umlsystem}[x=3, y=0]{Schedule API access}
\umlusecase[x=8, y=0, name=uc1, fill=red!30]{Sign up}
\umlusecase[x=6, y=0, name=uc2, fill=green!30]{Sign in}

\umlusecase[x=0, y=-2, name=uc3, fill=blue!1]{GET object}
\umlusecase[x=2.4, y=-2, name=uc31, fill=blue!1]{Dict}
\umlusecase[x=4, y=-2, name=uc32, fill=green!30]{User}
\umlusecase[x=6, y=-2, name=uc33, fill=red!30]{Admin}
\umlusecase[x=8, y=-2, name=uc34, fill=blue!1]{Schedule}

\umlusecase[x=0, y=-3, name=uc4, fill=green!30]{POST object}
\umlusecase[x=2.4, y=-3, name=uc41, fill=red!30]{Dict}
\umlusecase[x=4, y=-3, name=uc42, fill=red!30]{User}
\umlusecase[x=6, y=-3, name=uc43, fill=red!30]{Admin}
\umlusecase[x=8, y=-3, name=uc44, fill=green!30]{Schedule}

\umlusecase[x=0, y=-4, name=uc5, fill=green!30]{PUT object}
\umlusecase[x=2.4, y=-4, name=uc51, fill=red!30]{Dict}
\umlusecase[x=4, y=-4, name=uc52, fill=green!30]{User}
\umlusecase[x=6, y=-4, name=uc53, fill=red!30]{Admin}
\umlusecase[x=8, y=-4, name=uc54, fill=green!30]{Schedule}


\umlusecase[x=0, y=-5, name=uc6, fill=green!30]{DEL object}
\umlusecase[x=2.4, y=-5, name=uc61, fill=red!30]{Dict}
\umlusecase[x=4, y=-5, name=uc62, fill=red!30]{User}
\umlusecase[x=6, y=-5, name=uc63, fill=red!30]{Admin}
\umlusecase[x=8, y=-5, name=uc64, fill=green!30]{Schedule}


\end{umlsystem}

\umlinherit{uc33}{uc3}
\umlinherit{uc43}{uc4}
\umlinherit{uc53}{uc5}
\umlinherit{uc64}{uc6}

\umlassoc{admin}{uc1}
\umlassoc{user}{uc2}
\umlassoc{admin}{uc2}
\umlassoc{unreg}{uc3}
\umlassoc{user}{uc3}
\umlassoc{admin}{uc34}
\umlassoc{user}{uc4}
\umlassoc{admin}{uc44}
\umlassoc{user}{uc5}
\umlassoc{admin}{uc54}
\umlassoc{user}{uc6}
\umlassoc{admin}{uc64}

\end{umlstyle}
}{Доступ на виконання запитів до системи}{fig:ApiAccess}

Для забезпечення конфіденційності при обміні даними використовується JSON Web Token. Роути, що обробляють реєстраційні та авторизаційні запити, представлено на рис.~\ref{fig:ApiAccess}.

JSON Web Token це стандарт токена доступу на основі JSON, стандартизованого в RFC 7519. Використовується для верифікації тверджень. JSON Web Token складається з трьох частин: заголовка, вмісту і підпису.

В корисному навантаженні зберігається будь-яка інформація, яку потрібно перевірити. Кожен ключ в корисному навантаженні відомий як «заява». Як і заголовок, корисне навантаження кодується в base64. Після отримання заголовку і корисного навантаження, обчислюється підпис.
