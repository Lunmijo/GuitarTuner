\anonsection{ВСТУП}

Головна причина необхідності налаштування музичних інструментів це те, що погано налаштований інструмент позбавляє граючого та слухачів можливості почути складну та приємну вуху мелодію. Адже неналаштований інструмент не зможе видавати звуки, подібні до тих, що були задумані автором тієї чи іншої композициї.

Актуальність даного дослідження полягає у тому, що кожному музиканту періодично потрібно налаштовувати свій інструмент. А якщо мова йде про гітару, то найкращим варіантом буде налаштовувати її кожного разу перед тим, як почати грати.

Існуючі аналоги не задовільняють потреби потенційних користувачів, заважають музикантам великою кількістю реклами та/або дають можливість повноцінно використовувати додаток лише післе його придбання. Також вони у багатьох випадках перевантажені зайвими функціями та анімацією, що не несе сенсового навантаження, а тільки позбавляє частини користувачів можливості використовувати доданок, адже додаткові функції та анімація потребують додаткових ресурсів девайсу.

Об'єктом дослідження є розробка Android-додатків та цифрова обробка аудіосигналів у цих додатках.

Предметом дослідження є створення гітарного тюнеру, що працює із використанням перетворення Фур'є.

Метою роботи є реалізація мобильного додатку, яке спросить процес налаштування музичних інструментів для музикантів.

Для реалізації мети роботи поставлено наступні завдання:
\begin{enumerate}
	\item вивчити теоретичний аппарат: основи сольфеджио, основи функціонального аналізу, цифрова обробка сигналів;

    \item вибрати кінцеву платформу для якої буде розроблятися додаток;

    \item проаналізувати існуючі аналогі додатку, їх плюси та мінуси;

    \item проаналізувати технології створення мобільних додатків;

    \item проаналізувати методи аналізу звукових сигналів;

    \item вивчити засоби обробки звуку у ОС Android;

    \item розглянути готові рішення для аналізу звукового потіку обраним методом;

    \item розробити на основі проведених аналізів вимоги до додатку;

    \item спроектувати та реалізувати обрані алгоритми згідно з вимогами;
    
    \item розробити мінімалістичний інтерфейс, що має інформацію про отримані з мікрофону ноти;

    \item роздивитися технології тестування та покрити додаток тестами у необхідному обсязі.
\end{enumerate}

