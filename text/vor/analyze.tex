\section{АНАЛІЗ ІСНУЮЧИХ АНАЛОГІВ ДОДАТКУ ТА ВИБІР СТЕКУ ТЕХНОЛОГІЙ}

Взагалі додатки поділяються на два основних види: веб-додатки та нативні.

Веб-додаток — це кліент-серверний додаток, у якому клієнт взаємодіє з сервером за допомогою браузеру, а роль серверу відіграє веб-сервер. Логіка додатку розподілена між сервером та клієнтом, зберігання данних, як правило, відбувається на серверній стороні, а обмін інформацією відбувається з використанням мережевих технологій. Одним із головних переваг такої архітектури є той факт, що додаток не залежить від конкретної операційної системи користувача.

Нативний додаток - це прикладна програма, що була розроблена для використання на конкретній платформі чи пристрої.

Одна з головних переваг нативних додатків це оптимізація під конкретну операційну систему, тому такі додатки працюють корректно та швидко. Також такі додатки можуть використовувати аппаратні частини пристроїв, наприклад, камеру, мікрофон, геолокацію та інше.

Для аналізу були обрані додатки з відкритим вихідним кодом, що спрощує розуміння принципів роботи того чи іншого додатку, а також дає можливість оцінити його переваги та недоліки з точки зору програмування.

\subsection{Існуючі веб-додатки гітарних тюнерів}

https://github.com/googlearchive/guitar-tuner

https://github.com/mducharme/Tuner

https://github.com/jbergknoff/guitar-tuner

https://github.com/in-the-keyhole/khs-guitar-tuner

https://github.com/theanam/webaudio-guitar-tuner

\subsection{Мобільні додатки гітарних тюнерів}
ios:
https://github.com/theoc/ios-tuner

https://github.com/catch-twenty-two/iOS-Guitar-Tuner

https://github.com/klawed/iOS-String-Tuner

android:
https://github.com/chRyNaN/Android-Guitar-Tuner

https://github.com/LeandroCoello/Android-Guitar-Tuner

http://www.java2s.com/Open-Source/Android_Free_Code/Development/mobile/mjhavens_Guitar_Tuner_Android_App.htm

https://github.com/demantz/WearGuitarTuner/blob/master/guitartunerlibrary/src/main/java/com/mantz_it/guitartunerlibrary/GuitarTuner.java

\subsection{Вибір стеку технологій для розробки додатку "Гітарний тюнер"}

Для розробки додатку була обрана мова програмування Java. Головним критерієм вибору була швидкість та простота розгортування додатку на операційній системі Android. Для Java існує Android фреймворк, що дозволяє комфортно працювати з Android-середовищем та використовувати усі його можливості. Додаток є нативним, тобто створеним для Android. 

У якості системи контролю версій був використаний Git та сайт GitHub. Головною метою використання системи контролю версій є можливість у разі пошкодження значної частини програми повернути минулу версію.

Нумерація версій релізів виконана згідно з ресурсом Semantic Versioning 2.0.0 \cite{preston-werner}

Текст курсової роботи оформлений із використанням текстового процесора LaTeX. Головною причиною такого вибору є зручність редагування форматування тексту та можливість зберігати вихідні файли разом із самою програмою на сайті GitHub. У якості редактору LaTeX та хмарного сховища було використано середовище Overleaf.
