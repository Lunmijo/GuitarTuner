\section{АНАЛІЗ ІСНУЮЧИХ АНАЛОГІВ ДОДАТКУ ТА ВИБІР СТЕКУ ТЕХНОЛОГІЙ}

Взагалі додатки поділяються на два основних види: веб-додатки та нативні. Веб-додаток — це кліент-серверний додаток, у якому клієнт взаємодіє з сервером за допомогою браузеру, а роль серверу відіграє веб-сервер. Логіка додатку розподілена між сервером та клієнтом, зберігання данних, як правило, відбувається на серверній стороні, а обмін інформацією відбувається з використанням мережевих технологій. Одним із головних переваг такої архітектури є той факт, що додаток не залежить від конкретної операційної системи користувача.

Нативний додаток — це прикладна програма, що була розроблена для використання на конкретній платформі чи пристрої.

Одна з головних переваг нативних додатків це оптимізація під конкретну операційну систему, тому такі додатки працюють корректно та швидко. Також такі додатки можуть використовувати аппаратні частини пристроїв, наприклад, камеру, мікрофон, геолокацію та інше.

Для аналізу були обрані додатки з відкритим вихідним кодом, що спрощує розуміння принципів роботи того чи іншого додатку, а також дає можливість оцінити його переваги та недоліки з точки зору програмування.

\subsection{Існуючі додатки гітарних тюнерів}

Гітарний тюнер може бути створений для будь-якої платформи. Для аналізу у даній роботі було обрано шість гітарних тюнерів для різних платформ (iOS, Android, Windows та веб).

Веб-додатки:

Javascript / HTML 5 Tuner (https://github.com/mducharme/Tuner
)  веб-додаток гітарного тюнеру, що використуває біквадратний фільтр для фільтрування високих частот та швидке перетворення Фур'є для обробки отриманого звуку.

Webaudio Guitar tuner (https://github.com/theanam/webaudio-guitar-tuner
) створено за допомогою JavaScript, функціонал — можливість прослухати кожну ноту у стандартному налаштуванні. 

Гітарні тюнери для операційної системи iOS:

iOS String Tuner (https://github.com/klawed/iOS-String-Tuner
) — гітарний тюнер, що використувує швидке перетворення Фур'є для аналізу отриманого звуку.

Гітарні тюнери, для операційної системи Android:

Android Guitar Tuner (https://github.com/chRyNaN/Android-Guitar-Tuner
) використовує yin pitch detection. Окрім знаходження частоти ноти, отриманої з мікрофону дає можливість користувачу прослухати еталонні ноти у малій октаві. Подробиці щодо октав та частот нот: \ref{octaves-info}

Coello Guitar Tuner (https://github.com/LeandroCoello/Android-Guitar-Tuner) створений з можливістю використовувати декілька алгоритмів для пошуку частоти ноти, отриманої з мікрофону: MPM (McLeod Pitch Method), AMDF (функція середньої величини різниці), YIN Pitch Tracking (y AUBIO та більш продуктивна версія, яка використовує FFT), алгоритм динамічного вейвлету.

Для операційної системи Windows: 

FFT Guitar Tuner (https://github.com/freetomik/FFT\_GuitarTuner) використовує для пошуку найбільшої частоти ноти Java-бібліотеку JTransforms, алгоритм швидкого перетворення Фур'є.

\addimg{img/tuners-comparison.png}{1}{Таблиця порівняння тюнерів}{fig:tuners-comparison}

На таблиці (мал.~\ref{fig:tuners-comparison}) наведено порівняльну таблицю характеристик усіх поданих у роботі аналогів розроблюваного гітарного тюнеру. 
З цієї таблиці можна зробити висновки, що найбільш поширеними є два алгоритми: алгоритм швидкого перетворення Фур'є та YIN алгоритм. Це пов'язано з тим, що для досягнення поставленої мети саме ці два алгоритми підходять щонайкраще.

\subsubsection{Існуючі фреймворки для розробки Android-додатків}

Xamarin — цей фреймворк дозволяє писати код на C\#. Є кросплатформовим, тому його можна використувати для розробки під iOS та Windows. Однією з головних переваг є підтримка тестування додатку у хмарі. \cite{xamarin}

React Native дозволяє створювати мобільні додатки, використовуючи мову програмування JavaScript. Має схожу архітектуру з веб-фреймворкм React, та дозволяє створювати мобільни додатки з багатим інтерфейсом, використовуючи декларативні компоненти. \cite{react-native}

Qt Mobile це фреймворк, що дозволяє писати мобільні додатки із використанням мови програмування C++. Для написання додатків використувується середовище Qt Creator. \cite{qt-mobile}

Adobe Phone Gap це фреймворк, що дозволяє використовувати HTML5, CSS та JavaScript для розробки мобільних додатків. Corona SDK — рушій для розробки 2D ігор. Для створення додатків у цьому рушії використовується мова програмування Lua.

Android SDK це набір інструментів для розробника, що створений та підтримується розробниками операційної системи Android — Google. Надає дуже широкі можливості для розробки мобільних додатків. Для програмування найбільш зручно використовувати інтегроване середовище розробки Android Studio. Мовою програмування є Java.
\cite{androiddevelopers} 

Для розробки мобільного додатку гітарного тюнеру було прийнято рішення використувати Android SDK, тому що цей фреймворк підтримується розробниками операційної системи та має широку спільноту прихильників. Також завдяки інтегровному середовищі розробки Android Studio, Android SDK дуже зручний у використанні у розробці мобільних додатків.

\subsection{Вибір стеку технологій для розробки додатку "Гітарний тюнер"}

Для створення гітарного тюнеру можна використовувати будь-яку мову програмування, але не на кожній з них алгоритм буде працювати однаково еффективно та швидко. Залежно від платформи розробки, можна використовувати че не використувати додаткові фреймворки та бібліотеки.

Головним критерієм вибору була швидкість та простота розгортування додатку на операційній системі, а також кількість користувачів платформи. Дуже важливо, щоб додаток гітарного тюнеру гітарист міг використати у будь-який момент часу: під час гри вдома, на вулиці чи на концерті. 

Операційна система iOS встановлена на близько 1 млрд пристроїв. Веб-додаток позбавляє можливості використовувати додаток у ситуації відсутності доступу до мережі Інтернет. Операційна система Android встановлена близько 2,5 млрд пристроїв в усьому світі \cite{androiddevelopers}, що майже у 3 рази більше за кількість Android-пристроїв. Смартфони на базі Android дуже поширені та завжди поруч у гітаристів, що є користувачами цієї операційної системи. Виходячи з цього, кінцевою платформою для розробки додатку справедливо обрати операційну систему Android.

Для розробки додатку була обрана мова програмування Java. Для Java існує Android фреймворк, що дозволяє комфортно працювати з Android-середовищем та використовувати усі його можливості. Створюваний додаток є нативним, тобто створеним для Android. Головними недоліками конкурентів є: у випадку з JavaScript, що використовується у ReactNative та Adobe Phone Gap - JS це однопоточна мова програмуваня, що використуває асинхронні запити: у випадку з гітарним тюнером нам потрібно використовувати декілька потоків та синхронність роботи цих потоків. Qt Creator використовує мову програмування C++, що є достатньо низькорівневою і недостатньо комфортною у використані для розробки Android додатків. Corona SDK у більшості випадків використовується для розробки рушіів. 

Саме тому стеком технологій для розробки додатку було обрано Java та Android SDK. У якості системи контролю версій був використаний Git та сайт GitHub. Головною метою використання системи контролю версій є можливість у разі пошкодження значної частини програми повернути минулу версію.

Нумерація версій релізів виконана згідно зі специфікацією Semantic Versioning~2.0.0~\cite{semver}.