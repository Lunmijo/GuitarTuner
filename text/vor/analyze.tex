\section{АНАЛІЗ ІСНУЮЧИХ АНАЛОГІВ ДОДАТКУ ТА ВИБІР СТЕКУ ТЕХНОЛОГІЙ}

% \subimport{sen/analyse/}{History.tex}

\subsection{Веб-додатки гітарних тюнерів}

https://github.com/googlearchive/guitar-tuner

https://github.com/mducharme/Tuner

https://github.com/jbergknoff/guitar-tuner

https://github.com/in-the-keyhole/khs-guitar-tuner

https://github.com/theanam/webaudio-guitar-tuner

\subsection{Мобільні додатки гітарних тюнерів}
ios:
https://github.com/theoc/ios-tuner

https://github.com/catch-twenty-two/iOS-Guitar-Tuner

https://github.com/klawed/iOS-String-Tuner

android:
https://github.com/chRyNaN/Android-Guitar-Tuner

https://github.com/LeandroCoello/Android-Guitar-Tuner

http://www.java2s.com/Open-Source/Android_Free_Code/Development/mobile/mjhavens_Guitar_Tuner_Android_App.htm

\subsection{Вибір стеку технологій для розробки додатку "Гітарний тюнер"}

для разработки использовалась java по причине кроссплатформенная в частности фреймворк android по причине писалось нативное приложение под андроид
система контроля версий гит
текст писался в латехе в облаке по имени overleaf