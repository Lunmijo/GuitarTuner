\section{СТВОРЕННЯ ANDROID-ДОДАТКУ "ГІТАРНИЙ ТЮНЕР"}

\subsubsection{Необхідність використання декількох потоків у додатку}

нам нужна многопоточность потому что в одном потоке графический интерфейс а в другом - жырная математика
..ссылки на куски из теоретической части..

\subsection{Отримання даних з мікрофону}

гуглить как это сделать в андроиде

\subsection{Реалізація перетворення Фур'є для Android-середовища}

просто ссылки на куски теории

\subsection{Знаходження найбільшої частоти ноти}

ссылки на куски теории и кусочки кода можн

\subsection{Порівняння отриманої частоти ноти з табличними}

кусок кода с хэш мапой и описание к этому

\subsection{Створення UI/UX та виведення на екран результатів}

\addimg{img/perfect-tone.jpg}{0.3}{Ідеальна нота}{fig:PerfectTone}

\subsubsection{Відхилення проаналізованої ноти від еталонної}
(рис.~\ref{fig:PerfectTone})
\lipsum[1]
\addimg{img/bad-note.jpg}{0.3}{Неточна нота}{fig:BadNote}
\lipsum[1]
нота отклоняется от эталонной из за недостаточного/излишнего натягивания струны (рис.~\ref{fig:BadNote})

