\section{СТВОРЕННЯ ANDROID-ДОДАТКУ "ГІТАРНИЙ ТЮНЕР"}

\subsection{Розробка вимог до додатку та чек-лісту тестування}

Для тестування додатку було обрано використувати саме чек-ліст, а не повноцінний тест план, тому що у розробці додатку приймав участь один розробник, і чек-ліст виступив у якості допоміжного інструмента, що пояснював би, що конкретно треба протестувати, без уточнення деталей.

\subsubsection{Вимоги до додатку}

TODO: ТЕКСТУС

\subsubsection{Чек-ліст тестування}

Для мануального тестування додатку було розроблено чек-ліст, що складається з двох колонок: короткий опис тесту та статус тесту (passed/failed). У цьому чек-лісті тестами було покрито користувацький інтерфейс додатку та його функціональні можливості.

Додаток "Гітарний тюнер" - Статус

Пункт тесту 1 ------------ passed

пункт тесту 2 ------------ passed

пункт тесту 3 ------------ passed


\subsection{Проектування та реалізація алгоритму швидкого перетворення Фур'є}

TODO: ТЕКСТУС

\subsection{Необхідність використання декількох потоків у додатку}

TODO: нам нужна многопоточность потому что в одном потоке графический интерфейс а в другом - жырная математика
..ссылки на куски из теоретической части..

\subsection{Отримання даних з мікрофону}

Для отримання даних з мікрофону було створено пакет audiorecorder, у якому, відповідно, клас AndroidAudioRecorder.

\subsection{Реалізація перетворення Фур'є для Android-середовища}

\addCodeAsImg{\lstinputlisting[numbers=left]{code/MapFrequencyFinder.java}}{Хеш-таблиця нот}{fig:MapFrequencyFinder}
\addCodeAsImg{\lstinputlisting[numbers=left]{code/NoteName.java}}{Перерахування нот}{fig:NoteName}

\subsection{Знаходження найбільшої частоти ноти та порівняння її з табличними даними}

Для пошуку найбільшої частоти з отриманих даних після швидкого перетворення Фур'є було використано алгоритм прямого перебору. Для зберігання інформації про назви нот було обрано перелічувану структуру даних.

Перелічуваний тип даних - тип даних що складається з множини іменованих значень які називаються елементами, членами чи енумераторами типу.

У нотній нотації ми маємо 12 нот, а у переліченні - 13 (\ref{fig:NoteName}). 13 значення у переліченні це значення UNDEFINED, або "невідоме", на той випадок, якщо з якихось причин ми не можемо визначити висоту ноти, що була зчитана з даних мікрофону. 

Це тринадцяте значення є дуже важливим, адже допомагає уникнути помилок, пов'язаних із тим, що програма не може визначити знайдену ноту. У разі, коли значення ноти визначається як UNDEFINED, на екрані у користувача з'являється літера "U", що посвідчує про те, що програма не змогла визначити ноту.

Для зберігання інфомації про висоти нот було обрано структуру даних хеш-таблиця 
(асоціативний масив).

Хеш-таблиця — структура даних, що реалізує інтерфейс асоціативного масиву, а саме, вона дозволяє зберігати пари (ключ, значення) і здійснювати три операції: операцію додавання нової пари, операцію пошуку і операцію видалення за ключем.

У конструкторі класу MapFrequencyFinder ми ініціалізуємо приватну константну ("final", "фінальну") змінну що має тип хеш-таблиць із назвою ноти у якості ключу та Double значення еталонної частоти цієї ноти у якості значення (\ref{fig:MapFrequencyFinder}).



\subsection{Створення UI/UX та виведення на екран результатів}

Під час розробки користувацького інтерфейсу наголос був зроблений на те, щоб створювана програма була максимально простою та зрозумілою у використанні та максимально дружньою до користувача будь-якого рівня. Саме тому у програмі інформація про отриману ноту подана у англомовній нотації, що є стандартом серед музикантів усього світу. Також у програмі подана інформація про частоту знайденої ноти та найближчої до неї.

Для гітаристів-початківців подана інформація про те, що потрібно зробити зі струною - натягнути її сильніше (якщо знайдена нота нижча за очікувану) чи послабити її (якщо знайдена нота вища за очікувану). Це полегшує процес налаштування гітари гітаристами будь-якого рівня.



\subsubsection{Відхилення проаналізованої ноти від еталонної}

\addimg{img/perfect-tone.jpg}{0.4}{Ідеальна нота}{fig:PerfectTone}
\addimg{img/bad-note.jpg}{0.4}{Неточна нота}{fig:BadNote}

Якщо струна налаштована правильно (або достатньо близько за висотою до очікуваної), колір ноти на екрані смартфону або планшету стає зеленим (мал.~\ref{fig:PerfectTone}). 

Під нотою, що звучить або тільки що прозвучала є інформація про інші назви цієї ж ноти (їх дві). Ця інформація важлива, якщо гітара налаштовується не за стандартним налаштуванням "мі - ля - ре - соль - сі - мі", а наприклад у drop D ("пониження ре-дієз"), де замість ноти "мі" на шостій струні налаштовується нота "ре".

Якщо струна налаштована неправильно (або недостатньо близько за висотою до очікуваної), колір ноти на екрані смартфону або планшету стає червоним (мал.~\ref{fig:BadNote}).

На екрані з'являється прохання послабити або підтягнути струну відповідно до того, чи знайдена нота вища за еталонну, чи нижча. Також є інформація про те, яка частота ноти була зіграна користувачем программи.

TODO: дубликат-плйсхолдер Під нотою, що звучить або тільки що прозвучала є інформація про інші назви цієї ж ноти (їх дві). Ця інформація важлива, якщо гітара налаштовується не за стандартним налаштуванням "мі - ля - ре - соль - сі - мі", а наприклад у drop D ("пониження ре-дієз"), де замість ноти "мі" на шостій струні налаштовується нота "ре".

\subsection{Покриття додатку тестами}

TODO: для мануального тестирования додатка был разработан чек-лист состоящий из N пунктов;
автотестами покрыта часть функциональности с использованием фреймворка JUnit