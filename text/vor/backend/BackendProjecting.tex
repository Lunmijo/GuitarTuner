\subsection{Проектування та прототипування back-end частини}

Для обміну інформацією між користувацьким додатком та back-end частиною використовується протокол передачі гіпертекстових даних HTTP. Передачу даних забезпечує стек транспортних протоколів TCP/IP.
Одним з способів побудови мереживих HTTP-додатків є використання  асинхронного подієвого JavaScript–оточення Node. Для кожного з’єднання викликається функція зворотнього виклику, проте коли з’єднань немає Node засинає.

У Node не має функцій, що працюють напряму з I/O, тому процес не блокується ніколи. Як результат, на Node легко розробляти масштабовані системи \cite{zeiss2015node}.
Node широко використовує подієву модель, він приймає цикл подій за основу оточення, замість того, щоб використовувати його в якості бібліотеки. В інших системах відбувається блокування виклику для запуску циклу подій.

При розробці використано бібліотеку Express — гнучкий фреймворк для веб-застосунків, побудованих на Node.js, що надає широкий набір функціональності, полегшуючи створення надійних API.

Express забезпечує тонкий прошарок базової функціональності для веб-застосунків, що не спотворює звичну та зручну функціональність Node.js., при отриманні запиту він оброблюватиметься відповідно до визначення маршруту (рис.~\ref{fig:Route}), де app є екземпляром express, METHOD є методом HTTP-запиту, PATH є шляхом на сервері, HANDLER є функцією-обробником, що спрацьовує, коли даний маршрут затверджено як співпадаючий.

\addCodeAsImg{\lstinputlisting[numbers=left]{code/AppMethod.js}}{Структура визначення маршрутів}{fig:Route}

\subsubsection{Роути}

В процесі роботи використовується протокол прикладного рівня HTTP. Обмін повідомленнями йде за схемою «запит-відповідь». Для ідентифікації ресурсів HTTP використовує URI. 

В додатку обробляються основні  HTTP методи для взаємодії об’єктами (рис.~\ref{fig:ApiAccess}). 

Протокол HTTP не зберігає свого стану між парами «запит-відповідь». Компоненти, що використовують HTTP, можуть самостійно здійснювати збереження інформації про стан, пов'язаний з останніми запитами та відповідями. 

Одним з розповсюджених способів реалізації цього можна назвати так звані cookies — невеликі записи, що зберігаються браузером. Зазвичай, вони встановлюються при виконанні користувачем певних дій та надсилаються серверу разом з наступними запитами. 

На рис.~\ref{fig:AppSignUp} зображено процес створення адміністратором нового користувача системи.

\addCodeAsImg{% Auth sequence uml diagram

\documentclass[a4paper,10pt]{article}
\usepackage[english]{babel}
\usepackage[left=3cm, right=1cm, top=1cm, bottom=1cm]{geometry}

\usepackage{tikz-uml}

\sloppy
\hyphenpenalty 10000000

\begin{document}
\thispagestyle{empty}
\begin{center}
\begin{tikzpicture}
\begin{umlseqdiag}
	\umlactor[no ddots, x=1]{Admin}
	\umlboundary[no ddots, x=5]{App}
	\umldatabase[no ddots, x=14, fill=blue!20]{DB}
	
	\begin{umlcall}[op=Sign up request, type=synchron, return=Response, padding=3]{Admin}{App}
	
		\begin{umlfragment}[type=SignUp]
			\umlcreatecall[no ddots, x=8]{App}{DbUser}
			\begin{umlcall}[op=Init, type=synchron, return=Response]{App}{DbUser}
				\begin{umlcall}[op=Set data, type=synchron, return=]{DbUser}{DbUser}
					\umlcreatecall[no ddots, x=11]{DbUser}{DbRole}
					\begin{umlcall}[op=Set data, type=synchron, return=Role]{DbUser}{DbRole}
						\begin{umlcall}[op=Get all, type=synchron, return=List]{DbRole}{DB}\end{umlcall}
					\end{umlcall}
				\end{umlcall}
				
				\begin{umlfragment}[type=Creating, label=OK, fill=green!10]				
					\begin{umlcall}[op=Store, type=synchron]{DbUser}{DB}\end{umlcall}
					\begin{umlcall}[op=Success, type=synchron]{DbUser}{DbUser}\end{umlcall}
					\umlfpart[Error]				
					\begin{umlcall}[op=Error, type=synchron]{DbUser}{DbUser}\end{umlcall}
				\end{umlfragment}

			\end{umlcall}
		\end{umlfragment}
		
	\end{umlcall}
	
\end{umlseqdiag}
\end{tikzpicture}
\end{center}

\end{document}
}{Процес створення адміністратором нового користувача системи}{fig:AppSignUp}

Після отримання зазначеного POST запиту, сервер перевірить, чи має користувач відповідні права (блок Auth рис.~\ref{fig:CreateOperation}), створить об’єкт користувача з відповідними правами та збереже його в базі даних.

Створений користувач може входити до системи (рис.~\ref{fig:AppSignIn}) для виконання певних задач з користування системою.

\addCodeAsImg{% Auth sequence uml diagram

\documentclass[a4paper,10pt]{article}
\usepackage[english]{babel}
\usepackage[left=3cm, right=1cm, top=1cm, bottom=1cm]{geometry}

\usepackage{tikz-uml}

\sloppy
\hyphenpenalty 10000000

\begin{document}
\thispagestyle{empty}
\begin{center}
\begin{tikzpicture}
\begin{umlseqdiag}
	\umlactor[no ddots, x=1]{User}
	\umlboundary[no ddots, x=5]{App}
	\umldatabase[no ddots, x=14, fill=blue!20]{DB}
	
	\begin{umlcall}[op=sign in request, type=synchron, return=response, padding=3]{User}{App}
	
		\begin{umlfragment}[type=Sign in]
			\umlcreatecall[no ddots, x=8]{App}{DbUser}
				\begin{umlcall}[op=init, type=synchron, return=response]{App}{DbUser}
					\begin{umlcall}[op=find one, type=synchron, return=user]{DbUser}{DB}\end{umlcall}	
					
					\umlcreatecall[no ddots, x=11]{DbUser}{JWT}
					\begin{umlcall}[op=check password, type=synchron, return=response]{DbUser}{JWT}\end{umlcall}
					
					\begin{umlfragment}[type=validate, label=OK, fill=green!10]
						\begin{umlcall}[op=create token, type=synchron, return=token]{DbUser}{JWT}\end{umlcall}		
						\begin{umlcall}[op=success, type=synchron]{DbUser}{DbUser}\end{umlcall}
						\umlfpart[Error]				
						\begin{umlcall}[op=error, type=synchron]{DbUser}{DbUser}\end{umlcall}
					\end{umlfragment}
					
				\end{umlcall}	
		\end{umlfragment}
		
	\end{umlcall}
	
\end{umlseqdiag}
\end{tikzpicture}
\end{center}

\end{document}
}{Вхід користувача до системи}{fig:AppSignIn}

Можна зазначити, що адміністратор теж є користувачем, проте з особливими правами (диференціація на рис.~\ref{fig:Route}).

\subsubsection{Моделі}

В процесі проектування закладено серію моделей, що відповідають об’єктам предметної області. В рамках системи зберігаються в базі даних у вигляді таблиць з певними взаємозвязками (реляційну модель описано в підрозділі \ref{subsection:relationModel}). Для доступу до даних використовуються основні HTTP методи, що відповідають операціям CRUD (від Create, Read, Update, Delete), їх перелічено нижче.

\paragraph{GET}
\addCodeAsImg{% Auth sequence uml diagram

\documentclass[a4paper,10pt]{article}
\usepackage[english]{babel}
\usepackage[left=3cm, right=1cm, top=1cm, bottom=1cm]{geometry}

\usepackage{tikz-uml}

\sloppy
\hyphenpenalty 10000000

\begin{document}
\thispagestyle{empty}
\begin{center}
\begin{tikzpicture}
\begin{umlseqdiag}
	\umlactor[no ddots, x=1]{User}
	\umlboundary[no ddots, x=5]{App}
	\umldatabase[no ddots, x=14, fill=blue!20]{DB}
	
	\begin{umlcall}[op=get request, type=synchron, return=response, padding=3]{User}{App}
		\begin{umlfragment}[type=Auth, fill=cyan!20]
			\umlcreatecall[no ddots, x=8]{App}{JWT}
			\begin{umlcall}[op=init, type=synchron, return=response]{App}{JWT}
				\begin{umlcall}[op=verify JWT, type=synchron]{JWT}{JWT}\end{umlcall}
			\end{umlcall}
		\end{umlfragment}
		
		\begin{umlfragment}[type=Read, label=OK, fill=green!20]
	
			\umlcreatecall[no ddots, x=11]{App}{Object}
			\begin{umlcall}[op=parameters, type=synchron, return=object]{App}{Object}
				\begin{umlcall}[op=select query, type=synchron, return=rows]{Object}{DB}\end{umlcall}
					
			\end{umlcall}	
			
			\umlfpart[Error]
			
			\begin{umlcall}[op=error, type=synchron]{App}{App}\end{umlcall}
		
		\end{umlfragment}
	\end{umlcall}
		
	
\end{umlseqdiag}
\end{tikzpicture}
\end{center}

\end{document}
}{Виконання запиту на отримання об’єкта}{fig:ReadOperation}

Запитує вміст вказаного ресурсу, який може приймати параметри, що передаються в URI (рис.~\ref{fig:ReadOperation}). Згідно зі стандартом, ці запити є ідемпотентними — багатократне повторення одного і того ж запиту GET приводить до однакових результатів (за умови, що сам ресурс не змінився за час між запитами).

В запропонованій реалізації запит GET має дві версії — з параметром (ID) та без нього. Останній виконує дію (надає користувачу) не до конкретного об’єкту, а до всієї множини, що є необхідним в певних ситуаціях (наприклад, відображення списку всіх викладачів за певним критерієм).

\paragraph{HEAD}

Аналогічний GET, за винятком того, що у відповіді сервера відсутнє тіло. Це може бути необхідно для отримання мета-інформації.

\paragraph{POST}
\addCodeAsImg{\begin{umlstyle}

\begin{umlseqdiag}
	\umlactor[no ddots, x=1]{User}
	\umlboundary[no ddots, x=5]{App}
	\umldatabase[no ddots, x=14, fill=blue!20]{DB}
	
	\begin{umlcall}[op=post request, type=synchron, return=response, padding=3]{User}{App}
		\begin{umlcall}[op=auth procedure, type=synchron]{App}{App}\end{umlcall}
		
		\begin{umlfragment}[type=create, fill=green!20, label=OK]
				\umlcreatecall[no ddots, x=11]{App}{Object}
				\begin{umlcall}[op=init, type=synchron, return=object]{App}{Object}
					\begin{umlfragment}[type=Store, name=Store, fill=cyan!20, label=OK]
						\begin{umlcall}[op=insert query, type=synchron, return=result]{Object}{DB}
							\begin{umlcall}[op=run query, type=synchron]{DB}{DB}\end{umlcall}				
						\end{umlcall}
					\end{umlfragment}
				\end{umlcall}	
				
			\umlnote[x=15.1, y=-8.8, fill=cyan!20]{Store}{Check permissions for run query, validate and run it (store object or raise an exception)}
			\umlfpart[Error]
			\begin{umlcall}[op=undo creation, type=synchron,]{App}{Object}\end{umlcall}
			\begin{umlcall}[op=error, type=synchron]{App}{App}\end{umlcall}
		
		\end{umlfragment}
	\end{umlcall}
		
	
\end{umlseqdiag}

\end{umlstyle}

}{Виконання запиту на створення з аутентифікацією}{fig:CreateOperation}

Передає дані (наприклад, з форми на веб-сторінці) заданому ресурсу. При цьому передані дані включаються в тіло запиту. На відміну від методу GET, метод POST не є ідемпотентним, тобто багатократне повторення одних і тих же запитів POST може повертати різні результати (рис.~\ref{fig:CreateOperation}).

На першому етапі відбувається перевірка доступу користувача это створення об’єкту цього типу (авторизація), відповідно до прав доступу (рис.~\ref{fig:ApiAccess}).

\paragraph{PUT}
\addCodeAsImg{\begin{umlstyle}

\begin{umlseqdiag}
	\umlactor[no ddots, x=1]{User}
	\umlboundary[no ddots, x=5]{App}
	\umldatabase[no ddots, x=14, fill=blue!20]{DB}
	
	\begin{umlcall}[op=path request, type=synchron, return=sesponse, padding=3]{User}{App}
		\begin{umlcall}[op=auth procedure, type=synchron]{App}{App}\end{umlcall}
		
		\begin{umlfragment}[type=Update, label=OK, fill=green!20]
				\umlcreatecall[no ddots, x=11]{App}{Object}
				\begin{umlcall}[op=parameters, type=synchron, return=object]{App}{Object}
					\begin{umlcall}[op=select query, type=synchron, return=rows]{Object}{DB}\end{umlcall}
					\begin{umlcall}[op=change, type=synchron]{Object}{Object}\end{umlcall}
					\begin{umlcall}[op=store, type=synchron, return=result]{Object}{DB}\end{umlcall}
				\end{umlcall}	
			
			\umlfpart[Error]
			
			\begin{umlcall}[op=error, type=synchron]{App}{App}\end{umlcall}
		
		\end{umlfragment}
	\end{umlcall}
		
	
\end{umlseqdiag}

\end{umlstyle}
}{Виконання запиту на модифікацію існуючого об’єкту}{fig:UpdateOperation}

Завантажує вказаний ресурс на сервер. В розроблюваній системі використовується для редагування існуючих даних (рис.~\ref{fig:UpdateOperation}). 

В процесі виконання, спочатку з бази даних силами ORM вибирається конкретний об’єкт, в нього вносяться зміни, після чого він записується до сховища на заміну попередньої версії.

\paragraph{PATCH}

Завантажує частину ресурсу на сервер. При розробці необхідності у використанні не знайдено.

\paragraph{DELETE}
\addCodeAsImg{\begin{umlstyle}

\begin{umlseqdiag}
	\umlactor[no ddots, x=1]{User}
	\umlboundary[no ddots, x=5]{App}
	\umldatabase[no ddots, x=14, fill=blue!20]{DB}
	
	\begin{umlcall}[op=delete request, type=synchron, return=sesponse, padding=3]{User}{App}
		\begin{umlcall}[op=auth procedure, type=synchron]{App}{App}\end{umlcall}
		
		\begin{umlfragment}[type=Delete, label=OK, fill=green!20]
				\umlcreatecall[no ddots, x=11]{App}{Object}
				\begin{umlcall}[op=parameters, type=synchron, return=object]{App}{Object}
					\begin{umlcall}[op=select query, type=synchron, return=rows]{Object}{DB}\end{umlcall}
					\begin{umlcall}[op=set timestamp, type=synchron]{Object}{Object}\end{umlcall}
					\begin{umlcall}[op=store, type=synchron, return=result]{Object}{DB}\end{umlcall}
				\end{umlcall}	
				
				
			\umlfpart[Error]
			
			\begin{umlcall}[op=error, type=synchron]{App}{App}\end{umlcall}
		
		\end{umlfragment}
		
	\end{umlcall}
		
	\umlnote[x=8, y=-5.25, fill=cyan!20]{Object}{Record don't deletes really, but deleting timestamp sets to current}
	
\end{umlseqdiag}

\end{umlstyle}
}{Виконання запиту на видалення об’єкту}{fig:DeleteOperation}

Видаляє вказаний ресурс.
Слід звернути увагу, що в процесі виконання запиту на видалення об’єкту в системі, видалення як такого не відбувається. Замість цього в окреме поле таблиці вноситься інформація про час виконання цієї процедури (рис.~\ref{fig:DeleteOperation}).

Такий спосіб реалізації дозволяє з однієї сторони приховати дані, відмічені як видалені від подальшого використання, а з іншої — зберегти їх там, де вони вже використовуються. В іншому випадку, у зв’язку з реляційністю бази потрібно було б вирішувати дилему — або проводити циклічне видалення для збереження цілісності даних, втрачаючи всі об’єкти, що посилаються на той, що видаляється; або ускладнювати структури даних, що потенційно призведе до дублювання даних.

\subsubsection{Публічне API}
\addimg{QRcode.png}{0.25}{Приклад QR-коду з посиланням}{fig:QRcode}

В процесі проектування створено структуру роутів, котра може використовуватися сторонніми сервісами, у тому числі — і без авторизації в системі, що дозволяє отримувати інформацію про розклади власними силами для подальшого використання тим чи іншим чином. 

Також було проаналізовано перспективи при використанні QR-кодів (рис.~\ref{fig:QRcode}) з метою супроводження традиційного паперового розкладу (та інших документів), що публікується на стендах.

Хоча термін «QR code» є зареєстрованим товарним знаком японської корпорації «DENSO Corporation», їх використання не обкладається ніякими ліцензійними відрахуваннями, коди описані та опубліковані як стандарти ISO \cite{воронкін2014можливості}. Основна перевага QR-коду – легке розпізнавання скануючим обладнанням (за допомогою мобільного телефону, планшета або ноутбука з камерою, на яких встановлена програма для зчитування кодів, тощо).

Одним з способів використання QR-кодів в навчальному процесі, крім запропонованих (зокрема, задля забезпечення швидкого доступу до навчально-методичного забезпечення, довідкової літератури, веб-сервісів навчального закладу) \cite[146]{воронкін2014можливості},  можна назвати надання доступу до електронної версії розкладу.
