\section{ПЕРЕТВОРЕННЯ ФУР'Є У КОНТЕКСТІ ЦИФРОВОЇ ОБРОБКИ СИГНАЛІВ}

\subsection{Витяг з музикального теоретичного матеріалу}

\subsubsection{Види нот. Октави}

ноты бывают разной высоты и длительности. ноты могут звучать по отдельности и несколько сразу - это аккорды. ноты бывают диссонансные и консонансные (относительно друг друга).

инструменты нужно настраивать потому что плохо настроенный инструмент звучит для прошаренного слушателя абсолютно диссонансно и вообще не настраивать инструмент перед игрой богомерзко

октавы бывают разные
основные это малая, большая, первая, вторая, третья, четвертая - дальше на гитаре не взять никак ибо струна прилетит в глаз

\subsubsection{Частота нот у математичному сенсі}

с точки зрения математики частота ноты это частота колебаний волны

а звуковой сигнал это волна

Волна- это колебания, распространяющиеся в пространстве в течениие времени. (строчка копипасты)

волны бывают разные (описать какие конкретно волны бывают)

\subsubsection{Амплітуда та частота ноти}

амплитуда это наибольшее отклонение от начальной позиции

(копипаста википедии)

Амплитуда характеризует громкость звука. Частота определяет тон, высоту.

(/копипаста википедии)

\subsection{Обертони}

обертона определяют тембр звука

тембр это уникальное звучание 

у человека есть тембр голоса, у инструментов - тембр инструмента

в струнных музыкальных инструментах тембр получается за счет обертонов

(копипаста википедии)

обертоны это призвуки входящие в спектр (распределение значений физической величины) музыкального звука, высота обертонов выше высоты основного тона

Обертоны бывают гармоническими и негармоническими. Частоты гармонических обертонов больше частоты основного тона в 2, 3, 4, 5 и т. д. раз (кратность равна натуральному числу). Гармонические обертоны вместе с основным тоном называются гармониками и образуют натуральный звукоряд (/копипаста википедии)

сюда натыкать картиночек со спектром сигнала, обертонами и нотной записью

\subsection{Загальні положення про цифрову обробку сигналів}

\subsubsection{Аналоговий та цифровий сигнал.}

картиночки аналогового и цифрового сигнала, нужно в джупайтере отрисовать графики

аналоговый сигнал непрерывен во времени и аналоговый сигнал природный

цифровой сигнал это дискретизированный сигнал, то есть цифровой сигнал получается если аналоговый взять в течении времени в точках

вне имеющихся точек цифровой сигнал не определен

\subsubsection{Види спотворення сигналу та способи фільтрації сигналу}

нужно погуглить про это

\subsubsection{Принцип роботи мікрофону у смартфоні на базі операційної системи Android}

погуглить

\subsubsection{Види перетворення Фур'є}

для аналогового сигнала есть непрерывное фурье и оно ок только в математике потому что ой извините там интеграл на минус бесконечность до плюс бесконечности
а еще есть куча видов сверху

\subsection{Математичний аппарат перетворення Фур'є.}

погуглить и высрать

\subsubsection{Дискретизація вхідного сигналу за часом}

берем число отсчетов сигнала какое нравится, теорема найквиста-шеннона тут играет роль

\subsubsection{Виконання перетворення Фур'є для дискретизованого за часом сигналу}

как это можно сделать - погуглить

\subsubsection{Пошук найбільшої частоти серед дискретизованих точок сигналу}

по определенному правилу - погуглить - выбираем частоту и сравниваем с табличной дабы проверить является ли полученное число какой-то нотой или нет

\subsection{Багатопоточність у Java \& Android}

многопоточность конкретно в джаве потому что в абстрактных языках она нас сейчас не интересует

..ссылки на куски из практической части..

\subsubsection{Способи вирішення конфлікту між потоками}

погуглить про способы решения проблем между потоками, состояние гонки описать и т.п.




